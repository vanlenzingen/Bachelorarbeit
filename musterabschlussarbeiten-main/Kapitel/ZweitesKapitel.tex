%!TeX root = ./../MusterAbschlussarbeit.tex

%##########################################################
% Inhalt
%##########################################################

\clearpage
\chapter{Theoretische Grundlagen}
Inhalt des zweiten Kapitels (im Allgemeinen):
\begin{itemize}
    \item Schaffen Sie die theoretischen Grundlagen, welche notwendig sind um Ihre Lösung und deren Mehrwert zu verstehen
    \item Die Tiefe ist an Personen gerichtet, die fachlich zugeordnet sind, aber u.U. kein Fachwissen besitzen:
    \begin{itemize}
        \item Informatiker sind mit dem Programmieren vertraut: Sie brauchen also nicht die Programmiersprache X erklären; jedoch wird nicht jede Person einordnen können was Sie damit machen
    \end{itemize}
    \item Erarbeiten Sie Wissen was im dritten Kapitel als Grundlage für Ihr Konzept dient $\rightarrow$ es ist schwer etwas zu konzipieren, wenn Sie nicht wissen wovon Sie reden
\end{itemize}

\section{Maschine Learning}
Was genau ist mashine learning seit wann gibt es dieses etc pp

\section{Neuronale Netze}
Grobe funktionsweise neuronale netze Trainign etc

\section{Reinforcement Learning}
was genau sit reinforcement learning


\section{Agenten}
Warum agenten was machen sachen mit agenten dies das jenes

\section{PPO}
Hier sollte ienes Übersicht des PPO alghorhitmus stehen

\section{SAC}
Hier sollte ienes Übersicht des SAC alghorhitmus stehen

