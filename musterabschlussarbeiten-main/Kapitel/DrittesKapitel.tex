%!TeX root = ./../MusterAbschlussarbeit.tex

%##########################################################
% Inhalt
%##########################################################
\clearpage
\chapter{Konzeption}

\section{Anforderungsanalyse}
Einfache Implementierung
Visualisierung um Agent besser überwachen zu können
Frameworks für RL 
Frameworks für Überwachung

\section{Auswahl Entwicklungsumgebung}

\section{Methodik}

In diesem Abschnitt wird der methodische Ansatz zur Implementierung des Spiels und des Trainings des Reinforcement Learning-Agenten beschrieben.


\subsection{Implementierung des Spiels}

Das Spiel 'Noch mal' wurde mithilfe der Unity Engine implementiert, um eine interaktive und visuell ansprechende Umgebung zu schaffen. Die Programmierung der Spielregeln, die Benutzerinteraktion sowie die visuelle Darstellung des Spielzustands wurden dabei in Unity realisiert.

\subsection{Entwicklung des Reinforcement Learning-Agenten}

Der Reinforcement Learning-Agent wurde mithilfe der Unity ML-Agents-Bibliothek entwickelt, um das Spiel 'Noch mal' zu erlernen und zu spielen. ML-Agents bietet eine umfassende Sammlung von Werkzeugen und Algorithmen für die Implementierung von RL-Agenten in Unity.

\subsection{Trainingsprozess und Anpassungen}

Der Agent wurde in der implementierten Unity-Umgebung trainiert, wobei mehrere Trainingsdurchläufe durchgeführt wurden. Während des Trainingsprozesses wurden kontinuierliche Anpassungen vorgenommen, einschließlich der Feinabstimmung der Hyperparameter, der Modifikation der Netzwerkarchitektur und der Einführung von Maßnahmen zur Förderung der Exploration. Diese Anpassungen wurden iterativ durchgeführt, um die Leistung und die Lernfähigkeit des Agenten zu verbessern und sicherzustellen, dass er die Spielumgebung optimal erfasst und Strategien entwickelt, um die gestellten Ziele zu erreichen.

\subsection{Überwachung und Analyse}
Der Fortschritt des Trainings wurde kontinuierlich überwacht und analysiert. Dies umfasste die Überprüfung von Trainingsmetriken wie der durchschnittlichen Belohnung, der Verfolgung der Lernkurven und der Analyse von Trainingsfehlern.

\subsection{Iteration und Abschluss}
Der Trainingsprozess wurde iterativ durchgeführt, wobei der Agent kontinuierlich verbessert wurde, bis eine zufriedenstellende Leistung erreicht wurde. Nach Abschluss des Trainings wurde der finale Agent auf seine Fähigkeit getestet, das Spiel 'Noch mal' zu spielen, und seine Leistung wurde bewertet.
Dieser methodische Ansatz ermöglichte es, einen effektiven Reinforcement Learning-Agenten für das Spiel 'Noch mal' zu entwickeln und zu trainieren, der in der Lage ist, das Spiel auf einem angemessenen Niveau zu spielen.