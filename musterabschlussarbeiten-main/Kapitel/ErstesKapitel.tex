%!TeX root = ./../MusterAbschlussarbeit.tex

%##########################################################
% Inhalt
%##########################################################
\pagenumbering{arabic}
\chapter{Einleitung}

In der heutigen Zeit stehen wir an der Schwelle einer digitalen Revolution, in der maschinelles Lernen und künstliche Intelligenz eine immer bedeutendere Rolle spielen. Insbesondere das Gebiet des Reinforcement Learning hat in den letzten Jahren enorme Fortschritte gemacht und findet Anwendung in einer Vielzahl von Bereichen, von der Robotik bis hin zu Finanzen. Diese Entwicklung bietet aufregende Möglichkeiten, komplexe Probleme zu lösen und intelligente Systeme zu entwickeln, die in der Lage sind, eigenständig zu lernen und Entscheidungen zu treffen.

Reinforcement Learning (RL) hat bereits beeindruckende Erfolge erzielt, indem es Algorithmen entwickelt hat, die komplexe Spiele wie Go auf einem kompetitiven Niveau spielen können und sogar über menschliche Fähigkeiten hinausgehen. Darüber hinaus wurde RL erfolgreich eingesetzt, um die Effizienz und Leistung von Serverfarmen bei Unternehmen wie Google zu optimieren. Diese Anwendungen verdeutlichen die Vielseitigkeit und Leistungsfähigkeit von Reinforcement Learning bei der Bewältigung verschiedenster Herausforderungen und unterstreichen seine Fähigkeit, komplexe Probleme zu lösen und neue Lösungswege zu finden.

In dieser Bachelorarbeit liegt der Fokus auf der Implementierung und dem Training eines Reinforcement Learning-Agenten für das Würfelspiel 'Noch mal'. 'Noch mal' ist ein Würfelspiel, das Strategie und Glück erfordert. Ziel dieser Arbeit ist es, einen Agenten zu entwickeln, der in der Lage ist, das Spiel zu erlernen und auf einem kompetitiven Niveau zu spielen.

\newpage
Der Einsatz von Reinforcement Learning zur Bewältigung komplexer Spiele wie 'Noch mal' bietet eine interessante Herausforderung und die Möglichkeit, die Leistungsfähigkeit dieser Techniken zu demonstrieren. Durch die Implementierung eines RL-Agenten für dieses Spiel lässt sich untersuchen, wie gut maschinelle Lernmodelle in der Lage sind, komplexe Entscheidungsprobleme zu lösen und Strategien zu entwickeln, um ein definiertes Ziel zu erreichen.

Die Struktur dieser Arbeit gliedert sich wie folgt:
\setlist{noitemsep}
\begin{itemize}

    \item \textbf{Theoretische Grundlagen des Reinforcement Learning:} In diesem Abschnitt werden die grundlegenden Konzepte des Reinforcement Learning erläutert, darunter Markov-Entscheidungsprozesse, Politiken, Wertfunktionen und verschiedene RL-Algorithmen.
    \item \textbf{Das Würfelspiel 'Noch mal':} Hier wird eine detaillierte Beschreibung des Spiels 'Noch mal' gegeben, einschließlich der Regeln, Spielziele und möglicher Strategien.
    \item \textbf{Methodik:} Dieser Abschnitt beschreibt den Ansatz zur Implementierung des Reinforcement Learning-Agenten für 'Noch mal'. Hier werden die Wahl des RL-Algorithmus, die Modellarchitektur und die Trainingsparameter erläutert.
    \item \textbf{Experimente und Ergebnisse:} Es werden die Ergebnisse der Experimente präsentiert, einschließlich der Leistung des RL-Agenten beim Spielen von 'Noch mal' und einer Analyse seiner Fähigkeiten und Schwächen.
    \item \textbf{Diskussion:} Eine Diskussion über die Ergebnisse, die Einschränkungen der Studie und mögliche Verbesserungen wird vorgenommen.
    \item \textbf{Fazit und Ausblick:} Abschließend werden die wichtigsten Erkenntnisse zusammengefasst und ein Ausblick auf mögliche zukünftige Forschungsrichtungen gegeben.
\end{itemize}

Diese Bachelorarbeit zielt darauf ab, einen Beitrag zum Verständnis der Anwendung von Reinforcement Learning in der Spieleentwicklung zu leisten und Einblicke in die Leistungsfähigkeit dieser Techniken zu bieten. Durch die Implementierung eines RL-Agenten für 'Noch mal' sollen neue Erkenntnisse darüber gewonnen werden, wie maschinelles Lernen zur Entwicklung intelligenter Systeme in spielerischen Umgebungen eingesetzt werden kann.


\newpage

\section{Methodik}

In diesem Abschnitt wird der methodische Ansatz zur Implementierung des Spiels und des Trainings des Reinforcement Learning-Agenten beschrieben.


\subsection{Visualisierung}
\subsection{Implementierung des Spiels}

Das Spiel 'Noch mal' wurde mithilfe der Unity Engine implementiert, um eine interaktive und visuell ansprechende Umgebung zu schaffen. Die Programmierung der Spielregeln, die Benutzerinteraktion sowie die visuelle Darstellung des Spielzustands wurden dabei in Unity realisiert.

\subsection{Entwicklung des Reinforcement Learning-Agenten}

Der Reinforcement Learning-Agent wurde mithilfe der Unity ML-Agents-Bibliothek entwickelt, um das Spiel 'Noch mal' zu erlernen und zu spielen. ML-Agents bietet eine umfassende Sammlung von Werkzeugen und Algorithmen für die Implementierung von RL-Agenten in Unity.

\subsection{Trainingsprozess und Anpassungen}

Der Agent wurde in der implementierten Unity-Umgebung trainiert, wobei mehrere Trainingsdurchläufe durchgeführt wurden. Während des Trainingsprozesses wurden kontinuierliche Anpassungen vorgenommen, einschließlich der Feinabstimmung der Hyperparameter, der Modifikation der Netzwerkarchitektur und der Einführung von Maßnahmen zur Förderung der Exploration. Diese Anpassungen wurden iterativ durchgeführt, um die Leistung und die Lernfähigkeit des Agenten zu verbessern und sicherzustellen, dass er die Spielumgebung optimal erfasst und Strategien entwickelt, um die gestellten Ziele zu erreichen.

\subsection{Überwachung und Analyse}

Der Fortschritt des Trainings wurde kontinuierlich überwacht und analysiert. Dies umfasste die Überprüfung von Trainingsmetriken wie der durchschnittlichen Belohnung, der Verfolgung der Lernkurven und der Analyse von Trainingsfehlern.

\subsection{Iteration und Abschluss}

Der Trainingsprozess wurde iterativ durchgeführt, wobei der Agent kontinuierlich verbessert wurde, bis eine zufriedenstellende Leistung erreicht wurde. Nach Abschluss des Trainings wurde der finale Agent auf seine Fähigkeit getestet, das Spiel 'Noch mal' zu spielen, und seine Leistung wurde bewertet.

Dieser methodische Ansatz ermöglichte es, einen effektiven Reinforcement Learning-Agenten für das Spiel 'Noch mal' zu entwickeln und zu trainieren, der in der Lage ist, das Spiel auf einem angemessenen Niveau zu spielen.
