%!TeX root = ./../MusterAbschlussarbeit.tex

%##########################################################
% Inhalt
%##########################################################
\pagenumbering{arabic}
\chapter{Einleitung}

In der heutigen Zeit stehen wir an der Schwelle einer digitalen Revolution, in der maschinelles Lernen und künstliche Intelligenz eine immer bedeutendere Rolle spielen. Insbesondere das Gebiet des Reinforcement Learning hat in den letzten Jahren enorme Fortschritte gemacht und findet Anwendung in einer Vielzahl von Bereichen, von der Robotik bis hin zu Finanzen. Diese Entwicklung bietet aufregende Möglichkeiten, komplexe Probleme zu lösen und intelligente Systeme zu entwickeln, die in der Lage sind, eigenständig zu lernen und Entscheidungen zu treffen.

Reinforcement Learning (RL) hat bereits beeindruckende Erfolge erzielt, indem es Algorithmen entwickelt hat, die komplexe Spiele wie Go auf einem kompetitiven Niveau spielen können und sogar über menschliche Fähigkeiten hinausgehen. Darüber hinaus wurde RL erfolgreich eingesetzt, um die Effizienz und Leistung von Serverfarmen bei Unternehmen wie Google zu optimieren. Diese Anwendungen verdeutlichen die Vielseitigkeit und Leistungsfähigkeit von Reinforcement Learning bei der Bewältigung verschiedenster Herausforderungen und unterstreichen seine Fähigkeit, komplexe Probleme zu lösen und neue Lösungswege zu finden.

In dieser Bachelorarbeit liegt der Fokus auf der Implementierung und dem Training eines Reinforcement Learning-Agenten für das Würfelspiel 'Noch mal'. 'Noch mal' ist ein Würfelspiel, das Strategie und Glück erfordert. Ziel dieser Arbeit ist es, einen Agenten zu entwickeln, der in der Lage ist, das Spiel zu erlernen und auf einem kompetitiven Niveau zu spielen.

\newpage
Der Einsatz von Reinforcement Learning zur Bewältigung komplexer Spiele wie 'Noch mal' bietet eine interessante Herausforderung und die Möglichkeit, die Leistungsfähigkeit dieser Techniken zu demonstrieren. Durch die Implementierung eines RL-Agenten für dieses Spiel lässt sich untersuchen, wie gut maschinelle Lernmodelle in der Lage sind, komplexe Entscheidungsprobleme zu lösen und Strategien zu entwickeln, um ein definiertes Ziel zu erreichen.


Diese Bachelorarbeit zielt darauf ab, einen Beitrag zum Verständnis der Anwendung von Reinforcement Learning in der Spieleentwicklung zu leisten und Einblicke in die Leistungsfähigkeit dieser Techniken zu bieten. Durch die Implementierung eines RL-Agenten für 'Noch mal' sollen neue Erkenntnisse darüber gewonnen werden, wie maschinelles Lernen zur Entwicklung intelligenter Systeme in spielerischen Umgebungen eingesetzt werden kann.