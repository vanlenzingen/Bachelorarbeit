%!TeX root = ./../MusterAbschlussarbeit.tex

%##########################################################
% Inhalt
%##########################################################
\pagenumbering{arabic}
\chapter{Einleitung}

In der heutigen Zeit stehen die Menschheit an der Schwelle einer digitalen Revolution, in der künstliche Intelligenz eine immer bedeutendere Rolle spielt. Insbesondere das Gebiet des Reinforcement Learnings hat in den letzten Jahren enorme Fortschritte gemacht und findet Anwendung in einer Vielzahl von Bereichen. Diese Entwicklung bietet vielfältige Möglichkeiten, komplexe Probleme zu lösen und intelligente Systeme zu entwickeln, die in der Lage sind, eigenständig zu lernen und Entscheidungen zu treffen. 

Reinforcement Learning (RL) hat beeindruckende Erfolge erzielt, indem es Algorithmen entwickelt hat, die komplexe Spiele wie 'Go', auf einem kompetitiven Niveau spielen können und sogar menschliche Fähigkeiten übersteigt Diese Anwendung verdeutlicht die Vielseitigkeit und Leistungsfähigkeit von Reinforcement Learning bei der Bewältigung verschiedenster Herausforderungen und unterstreichen die Fähigkeiten von RL , komplexe Probleme zu lösen und neue Lösungswege zu finden.\cite{hui_alphago_2018, noauthor_alphago_2020} 

In dieser Bachelorarbeit liegt der Fokus auf der Implementierung und dem Training eines Reinforcement Learning-Agenten für ein in Unity umgesetztes Würfelspiel. 'Noch mal' ist ein Würfelspiel, das Strategie und Glück erfordert.  Durch die Implementierung eines RL-Agenten für dieses Spiel, lässt sich untersuchen, wie gut maschinelle Lernmodelle in der Lage sind, komplexe Entscheidungsprobleme zu lösen und Strategien zu entwickeln, um ein definiertes Ziel zu erreichen. Diese Arbeit zielt darauf ab, einen Beitrag zum Verständnis der Anwendung von Reinforcement Learning in der Spieleentwicklung zu leisten und Einblicke in die Leistungsfähigkeit dieser Techniken zu bieten. 
\newpage
Die Arbeit gliedert sich in folgende Abschnitte:
\begin{itemize}[noitemsep]
\item \textbf{Theoretische Grundlagen:} Die theoretischen Grundlagen zum Reinforcement Learning und neuronalen Netzen werden hier dargestellt. Außerdem werden die Spielregeln des modellierten Spiels beschrieben.
\item \textbf{Konzeption:} Dieses Kapitel beinhaltet eine Anforderungsanalyse. Auf Grundlage dieser die Auswahl der verwendeten Technologien begründet.
\item \textbf{Implementierung:} In diesem Kapitel wird dargestellt, wie das Spiel und der Agent implementiert wurde.
\item \textbf{Präsentation der Ergebnisse:} Für diesen Abschnitt wurden verschiedene Versuche durchgeführt. Die Ergebnisse dieser werden in diesem Kapitel dargestellt. 
\item \textbf{Auswertung und Ausblick:} In diesem Kapitel werden die Ergebnisse bewertet und weitere Schritte zur Verbesserung des Agenten diskutiert.
\end{itemize}