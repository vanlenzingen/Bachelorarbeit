%!TeX root = ./../MusterAbschlussarbeit.tex


%##########################################################
% Pakete - Stil, Sprache und Schriftart
%##########################################################
\usepackage[UKenglish,ngerman]{babel}
\usepackage[T1]{fontenc}
\usepackage[osf,scale=1.15]{sourcesanspro}
\usepackage{csquotes}
\usepackage[backend=biber,sorting=none,bibencoding=utf8]{biblatex}
\addbibresource{MusterBachelorArbeitBib.bib}
\usepackage[headsepline]{scrlayer-scrpage}
\usepackage{fontspec}
\setcounter{tocdepth}{\subsubsectiontocdepth}
\addtokomafont{chapterentry}{\bfseries}
\usepackage[onehalfspacing]{setspace}
\usepackage{enumitem}
%## #### #### ####### ###
\usepackage[absolute, overlay]{textpos}
\usepackage{svg}
%\DeclareGraphicsExtensions{.png,.jpg,.gif,.pdf,.svg}
%\renewcommand\includesvg[2][]{\includegraphics{#2}}

\AtBeginDocument{%
	\providecaptionname{ngerman}{\lstlistlistingname}{Quellcodeverzeichnis}
	\providecaptionname{ngerman}{\lstlistingname}{Quellcode}
}

%##########################################################
% Pakete - Grafisches Elemente und Farben
%##########################################################
\usepackage{graphicx}
\usepackage{xcolor}

\providecommand{\keywords}[1]{\textbf{\textit{Keywords:}} #1}

%############ HTWK Farben #################
\xdefinecolor{htwkGelb}{rgb}{0.996,0.925,0}
\xdefinecolor{htwkGrau}{rgb}{0.945,0.945,0.945}
\xdefinecolor{htwkBlau}{rgb}{0,0.273,0.597}
\xdefinecolor{htwkMagenta}{rgb}{0.894,0,0.488}
\xdefinecolor{htwkRot}{rgb}{0.894,0.1875,0.035}
\xdefinecolor{htwkGruen}{rgb}{0,0.586,0.304}
\xdefinecolor{htwkCyan}{rgb}{0,0.617,0.886}
\xdefinecolor{codegreen}{rgb}{0,0.6,0}
%######## Legen Sie die Farbe fest #####
% htwkBlau, htwkGruen, htwkRot, htwkCyan, htwkGelb, htwkMagenta, htwkGrau
\newcommand{\farbe}{htwkBlau}
%#########################################

%Linkformatierung
\usepackage[
	    colorlinks=true,
	    urlcolor=htwkBlau,
	    citecolor=htwkBlau,
		pdftitle={\titel},
		pdfauthor={\autor},
		breaklinks=true,
		hidelinks
]{hyperref}

%##########################################################
% Pakete - Zusätzliche
%##########################################################

\usepackage{listings}
\usepackage{scrhack}
\usepackage{microtype}
\usepackage{acro}

%##########################################################
% Code Snippet Definitionen
%##########################################################
\newcommand\digitstyle{\color{purple}}
\makeatletter
\newcommand{\ProcessDigit}[1]
{%
  \ifnum\lst@mode=\lst@Pmode\relax%
   {\digitstyle #1}%
  \else
    #1%
  \fi
}

\lstdefinestyle{MyStyle}{
	basicstyle=\footnotesize\sffamily,
	keywordstyle=\color{blue},
	commentstyle=\color{codegreen},
	stringstyle=\color{orange!80!black},
	morecomment=[l]{//},
	morecomment=[l]{/*}{*/},
	morestring=[b]",
	morestring=[b]',
	breakatwhitespace=false,         
    breaklines=true,                 
    keepspaces=true,
	showspaces=false,                
    showstringspaces=false,
    showtabs=false,                  
    tabsize=2,
	numbers=left,
	numberstyle=\tiny,
	frameround=tttt,
	frame=trbl,
	columns=fullflexible,
	xleftmargin=0.03\linewidth,
	xrightmargin=0.01\linewidth,
	inputencoding=utf8,
}

\lstset{
	style=MyStyle,
	literate=%
    {0}{{{\ProcessDigit{0}}}}1
    {1}{{{\ProcessDigit{1}}}}1
    {2}{{{\ProcessDigit{2}}}}1
    {3}{{{\ProcessDigit{3}}}}1
    {4}{{{\ProcessDigit{4}}}}1
    {5}{{{\ProcessDigit{5}}}}1
    {6}{{{\ProcessDigit{6}}}}1
    {7}{{{\ProcessDigit{7}}}}1
    {8}{{{\ProcessDigit{8}}}}1
    {9}{{{\ProcessDigit{9}}}}1
}
