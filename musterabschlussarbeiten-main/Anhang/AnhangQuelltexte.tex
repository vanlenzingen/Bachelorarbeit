%!TeX root = ./../MusterAbschlussarbeit.tex

%##########################################################
% Inhalt
%##########################################################
\clearpage
\chapter{Anhang - Quelltexte}
\lstinputlisting[language=csh, label={lst:SetColor}, caption=Switch Case Statement zum Anpassen der Farben des Farbwürfels]{Programmcode/SetColor.cs}
\lstinputlisting[language=csh, label={lst:PushValidField}, caption=Befüllen des Beobachtungsvektor mit validem Feld]{Programmcode/PushValidField.cs}
\lstinputlisting[language=csh, label={lst:DiceShift}, caption=Erhöhen der Warscheinlichkeit eine bestimmte Farbe zu würfeln]{Programmcode/Würfelshift.cs}
\newpage
\lstinputlisting[language=csh, label={lst:CollectObservations}, caption=Erstellen des Beobachtungsvektors]{Programmcode/CollectObservations.cs}

\newpage
\lstinputlisting[language=python,label={lst:ImportLogfiles}, caption=Einlesen aller angegebenen Logdateien]{Programmcode/ImportLogfiles.py}
 \newpage
\lstinputlisting[language=python,label={lst:DisplayGraph}, caption=Methode zum Anzeigen des Graphen]{Programmcode/DisplayGraph.py}

\lstinputlisting[language=csh, label={lst:GetAvailableFields}, caption=Gibt alle validen Felder für das gewählte Würfelpaar zurück]{Programmcode/GetAvailableFields.cs}
\lstinputlisting[language=csh, label={lst:CalculateNeighbors}, caption=Gibt alle benachbarten Felder der selben Farbe zurück]{Programmcode/CalculateNeighbors.cs}
\lstinputlisting[language=csh, label={lst:PickField}, caption=Interpolation über alle möglichen Felder]{Programmcode/PickField.cs}