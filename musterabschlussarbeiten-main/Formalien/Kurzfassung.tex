%!TeX root = ./../MusterAbschlussarbeit.tex

%##########################################################
% Inhalt
%##########################################################

\clearpage
\chapter*{Kurzfassung}
Die Kurzfassung - präziser: Abstrakt, sollte auf eine Seite beschränkt sein. Der Abstrakt
sollte das Ziel der Arbeit explizit beschreiben, die angewandten \textit{Methoden} aufführen, die
wichtigsten \textit{Ergebnisse} aufzählen und die \textit{Hauptschlussfolgerungen} aufzeigen.

Einen Weg zu finden, hunderte von Seiten an Informationen in wenigen Sätzen zusammenzufassen
ist eine Herausforderung. Jedes Wort muss sorgfältig abgewogen werden
(Gibt es eine bessere, prägnantere Möglichkeit, die Hauptaussage auszudrücken?). Die
einzelnen Sätze sollten mit größter sorgfalt kombiniert werden.

Das Verfassen des Abstraktes sollte bis zum Schluss aufgeschoben werden, aus dem
selben Grund, warum der Titel der Arbeit erst dann endgültige Form erhalten sollte,
wenn das entsprechende Arbeit ansonsten vollständig ist.

Die nachfolgenden Ausführungen sollen nochmal ins Gedächtnis rufen was die Anforderungen an einzelne Abschnitte einer Abschlussarbeit
an der \ac{fim} der \ac{htwk} sind.

Die Kurzfassung schließt mit der Nennung von 5 bis 10 Schlagwörtern (Keywords) ab,
welche der Verschlagwortung bspw. für die Recherche nutzbar gemacht werden. Genauer gesagt, sind das inhaltliche Schlüsselwörter Ihrer Arbeit.\\\\



Die Struktur dieser Arbeit gliedert sich wie folgt:
\setlist{noitemsep}
\begin{itemize}

    \item \textbf{Theoretische Grundlagen:} In diesem Abschnitt werden die grundlegenden Konzepte des Reinforcement Learning erläutert, darunter Neuronale Netze Markov-Entscheidungsprozesse und Probleme, welche bei dem RL auftreten können. Weiterhin wird eine detaillierte Beschreibung des Spiels 'Noch mal' gegeben, einschließlich der Regeln, Spielziele und möglicher Strategien.
    \item \textbf{Konzeption:} Dieser Abschnitt umfasst eine Anforderungsanalyse und  beschreibt den Ansatz zur Implementierung des Reinforcement Learning-Agenten für 'Noch mal'. 
    \item \textbf{Experimente und Ergebnisse:} Es werden die Ergebnisse der Experimente präsentiert, einschließlich der Leistung des RL-Agenten beim Spielen von 'Noch mal' und einer Analyse seiner Fähigkeiten und Schwächen.
    \item \textbf{Diskussion:} Eine Diskussion über die Ergebnisse, die Einschränkungen der Studie und mögliche Verbesserungen wird vorgenommen.
    \item \textbf{Fazit und Ausblick:} Abschließend werden die wichtigsten Erkenntnisse zusammengefasst und ein Ausblick auf mögliche zukünftige Forschungsrichtungen gegeben.
\end{itemize}


\keywords{IoT, SD-WAN, \ac{ml}, \ac{ftth}}
